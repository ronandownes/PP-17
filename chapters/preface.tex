\chapter*{Preface}
\addcontentsline{toc}{chapter}{Preface}

 
 
I am of the opinion that every \LaTeX\xspace maths teacher , at least once during 
his career, feels the need to create his or her own resources : this is what 
happened to me and here is the result, which, however, should be seen as 
a work still in progress. 


% Actually, this class is not completely 
% original, but it is a blend of all the best ideas that I have found in a 
% number of guides, tutorials, blogs and tex.stackexchange.com posts. In 
% particular, the main ideas come from two sources:

% \begin{itemize}
% 	\item \href{https://3d.bk.tudelft.nl/ken/en/}{Ken Arroyo Ohori}'s 
% 	\href{https://3d.bk.tudelft.nl/ken/en/nl/ken/en/2016/04/17/a-1.5-column-layout-in-latex.html}{Doctoral 
% 	Thesis}, which served, with the author's permission, as a backbone 
% 	for the implementation of this class;
% 	\item The 
% 		\href{https://github.com/Tufte-LaTeX/tufte-latex}{Tufte-Latex 
% 			Class}, which was a model for the style.
% \end{itemize}

% The first chapter of this book is introductive and covers the most 
% essential features of the class. Next, there is a bunch of chapters 
% devoted to all the commands and environments that you may use in writing 
% a book; in particular, it will be explained how to add notes, figures 
% and tables, and references. The second part deals with the page layout 
% and design, as well as additional features like coloured boxes and 
% theorem environments.

% I started writing this class as an experiment, and as such it should be 
% regarded. Since it has always been indended for my personal use, it may 
% not be perfect but I find it quite satisfactory for the use I want to 
% make of it. I share this work in the hope that someone might find here 
% the inspiration for writing his or her own class.

 
 
 
My observations indicate that  the  renewal  and reform process  of the Irish Maths school curriculum is not yet complete 
and with the help of the emphatic encouragement  of my students    I finally decided not to put the writing of this book off any longer.  The
development of the revised Mathematics curriculum is based on the curriculum aims of
Mathematics education, guiding principles of curriculum design, and  a broadening in assessment
stipulated in Mathematics Education Key Learning  Curriculum Documents. 



It addresses the requirement that transferable skills be as much a part of the teaching and learning as the academic subject matter itself.
% This booklet is one of the series Supplement to Mathematics Education Key Learning
% Area Curriculum Guide (Primary 1 - Secondary 6) (2017), aiming at providing a
% detailed account of:
% 1. the learning targets of the junior secondary Mathematics curriculum;
% 2. the learning content of the junior secondary Mathematics curriculum; and
% 3. the flow chart showing the progression pathways for the learning units of junior
% secondary Mathematics curriculum.
% Comments and suggestions on this booklet are most welcomed. They may be sent to:
% Chief Curriculum Development Officer (Mathematics)
% Curriculum Development Institute
% Education Bureau
% 4/F, Kowloon Government Offices
% 405 Nathan Road, Kowloon
% Fax: 3426 9265
% E-mail: ccdoma@edb.gov.hk




\begin{flushright}
	\textit{Ronan Downes}
\end{flushright}




% Add the preface to the table of contents as a chapter

