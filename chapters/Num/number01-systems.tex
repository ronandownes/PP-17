
\label{chap:numbersystems}




\chapter{\protect\hyperlink{chap:\thechapter}{Numeral and Number systems}}
\addtocontents{toc}{\protect\hypertarget{chap:\thechapter}{}}


\section{Integers}
Maths is divided in two based on whether the quantities it describes are discrete or continuous.
\section{Fractions, decimals, percentages  }












\section{Counting Is Arithmetic}
You have probably already learned  how to count! You may be saying: “But I already know how to count: one, two, three, $\ldots$”

True. But most counting problems do not involve simply counting a list or group of items. Usually we have to first figure out what we’re counting, then we have to figure out how to count it.

One thing that we will repeat over and over in this chapter, and indeed in the course of the entire book, is:

\begin{claim}
Don’t memorize!
\end{claim}



\section{Defining thereoms}
Theorems can easily be defined
\newtheorem{prop}{Proposition}[section]
\begin{theorem}
Let $f$ be a function whose derivative exists in every point, then $f$ is 
a continuous function.
\end{theorem}

\begin{theorem}[Pythagorean theorem]
\label{pythagorean}
This is a theorema about right triangles and can be summarised in the next 
equation 
\[ x^2 + y^2 = z^2 \]
\end{theorem}

And a consequence of theorem \ref{pythagorean} is the statement in the next 
corollary.

\begin{corollary}
There's no right rectangle whose sides measure 3cm, 4cm, and 6cm.
\end{corollary}

You can reference theorems such as \ref{pythagorean} when a label is assigned.

\begin{lemma}
Given two line segments whose lengths are $a$ and $b$ respectively there is a 
real number $r$ such that $b=ra$.
\end{lemma}

\section{}




\section{}

\section{}



\section{}


\section{}







\section{}




\section{}








\section{}