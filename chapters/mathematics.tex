\setchapterstyle{kao}
\setchapterpreamble[u]{\margintoc}




\chapter{\protect\hyperlink{chap:\thechapter}{Mathematics and Boxes}}
\addtocontents{toc}{\protect\hypertarget{chap:\thechapter}{}}
\labch{mathematics}

\section{Theorems}

Despite most people complain at the sight of a book full of equations, 
mathematics is an important part of many books. Here, we shall 
illustrate some of the possibilities. We believe that theorems, 
definitions, remarks and examples should be emphasised with a shaded 
background; however, the colour should not be to heavy on the eyes, so 
we have chosen a sort of light yellow.\sidenote{The boxes are all of the 
same colour here, because we did not want our document to look like 
\href{https://en.wikipedia.org/wiki/Harlequin}{Harlequin}.}

\begin{definition}
\labdef{openset}
Let $(X, d)$ be a metric space. A subset $U \subset X$ is an open set 
if, for any $x \in U$ there exists $r > 0$ such that $B(x, r) \subset 
U$. We call the topology associated to d the set $\tau\textsubscript{d}$ 
of all the open subsets of $(X, d).$
\end{definition}

\refdef{openset} is very important. I am not joking, but I have inserted 
this phrase only to show how to reference definitions. The following 
statement is repeated over and over in different environments.

\begin{theorem}
A finite intersection of open sets of (X, d) is an open set of (X, d), 
i.e $\tau\textsubscript{d}$ is closed under finite intersections. Any 
union of open sets of (X, d) is an open set of (X, d).
\end{theorem}

\begin{proposition}
A finite intersection of open sets of (X, d) is an open set of (X, d), 
i.e $\tau\textsubscript{d}$ is closed under finite intersections. Any 
union of open sets of (X, d) is an open set of (X, d).\marginnote{You can even insert footnotes inside the theorem 
	environments; they will be displayed at the bottom of the box.}
\end{proposition}

\begin{lemma}
A finite intersection\footnote{I'm a footnote} of open sets of (X, d) is 
an open set of (X, d), i.e $\tau\textsubscript{d}$ is closed under 
finite intersections. Any union of open sets of (X, d) is an open set of 
(X, d).
\end{lemma}

You can safely ignore the content of the theorems\ldots I assume that if 
you are interested in having theorems in your book, you already know 
something about the classical way to add them. These example should just 
showcase all the things you can do within this class.

\begin{corollary}[Finite Intersection, Countable Union]
A finite intersection of open sets of (X, d) is an open set of (X, d), 
i.e $\tau\textsubscript{d}$ is closed under finite intersections. Any 
union of open sets of (X, d) is an open set of (X, d).
\end{corollary}

\begin{proof}
The proof is left to the reader as a trivial exercise. Hint: \blindtext
\end{proof}

\begin{definition}
Let $(X, d)$ be a metric space. A subset $U \subset X$ is an open set 
if, for any $x \in U$ there exists $r > 0$ such that $B(x, r) \subset 
U$. We call the topology associated to d the set $\tau\textsubscript{d}$ 
of all the open subsets of $(X, d).$\marginnote{
	Here is a random equation, just because we can:
	\begin{equation*}
  x = a_0 + \cfrac{1}{a_1
          + \cfrac{1}{a_2
          + \cfrac{1}{a_3 + \cfrac{1}{a_4} } } }
	\end{equation*}
}
\end{definition}

\begin{example}
Let $(X, d)$ be a metric space. A subset $U \subset X$ is an open set 
if, for any $x \in U$ there exists $r > 0$ such that $B(x, r) \subset 
U$. We call the topology associated to d the set $\tau\textsubscript{d}$ 
of all the open subsets of $(X, d).$
\end{example}

\begin{remark}
Let $(X, d)$ be a metric space. A subset $U \subset X$ is an open set 
if, for any $x \in U$ there exists $r > 0$ such that $B(x, r) \subset 
U$. We call the topology associated to d the set $\tau\textsubscript{d}$ 
of all the open subsets of $(X, d).$
\end{remark}

As you may have noticed, definitions, example and remarks have 
independent counters; theorems, propositions, lemmas and corollaries 
share the same counter.

\begin{remark}
Here is how an integral looks like inline: $\int_{a}^{b} x^2 dx$, and 
here is the same integral displayed in its own paragraph:
\[\int_{a}^{b} x^2 dx\]
\end{remark}

We provide two files for the theorem styles: 
\href{style/plaintheorems.sty}{plaintheorems.sty}, which you should 
include if you do not want coloured boxes around theorems; and 
\href{style/mdftheorems.sty}{mdftheorems.sty}, which is the one used for 
this document.\sidenote{The plain one is not showed, but actually it is 
exactly the same as this one, only without the yellow boxes.} Of course, 
you will have to edit these files according to your taste and the 
general style of the book.

\section[Boxes \& Environments]{Boxes \& Custom Environments
\sidenote[][*1.8]{Notice that in the table of contents and in the 
	header, the name of this section is \enquote{Boxes \& Environments}; 
	we achieved this with the optional argument of the \texttt{section} 
	command.}}

Say you want to insert a special section, an optional content or just 
something you want to emphasise. We think that nothing works better than 
a box in these cases. We used \Package{mdframed} to construct the ones 
shown below. You can create and modify such environments by editing the 
provided file \href{style/environments.sty}{environments.sty}.

\begin{kaobox}[frametitle=Title of the box]
\blindtext
\end{kaobox}

If you set up a counter, you can even create your own numbered 
environment.

\begin{kaocounter}
	\blindtext
\end{kaocounter}

\section{Experiments}

It is possible to wrap marginnotes inside boxes, too. Audacious readers 
are encouraged to try their own experiments and let me know the 
outcomes.

\marginnote[-2.2cm]{
	\begin{kaobox}[frametitle=title of margin note]
		Margin note inside a kaobox.\\
		(Actually, kaobox inside a marginnote!)
	\end{kaobox}
}

I believe that many other special things are possible with the 
\Class{kaobook} class. During its development, I struggled to keep it as 
flexible as possible, so that new features could be added without too 
great an effort. Therefore, I hope that you can find the optimal way to 
express yourselves in writing a book, report or thesis with this class, 
and I am eager to see the outcomes of any experiment that you may try.

%\begin{margintable}
	%\captionsetup{type=table,position=above}
	%\begin{kaobox}
		%\caption{caption}
		%\begin{tabular}{ |c|c|c|c| }
			%\hline
			%col1 & col2 & col3 \\
			%\hline
			%\multirow{3}{4em}{Multiple row} & cell2 & cell3 \\ & cell5 
			%%& cell6 \\ 
			%& cell8 & cell9 \\
			%\hline
		%\end{tabular}
	%\end{kaobox}
%\end{margintable}
